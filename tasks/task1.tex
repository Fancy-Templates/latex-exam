\begin{task}[8.5]{Eine spaßige Aufgabe}
Ein einfacher, allgemeiner, Einführungstext.

\begin{subtasks}
   % \item Hello % this would cause an error (as a safe-guard)
   \subtask{3} Dies ist eine Teilaufgabe die drei Punkte wert ist. Für die Antwort geben wir den Studierenden (dynamisch viel) freien Platz!
   \begin{examples}
      \item \texttt{1 + 1} sollte \texttt{2} ergeben
      \item \texttt{AnnA} ist ein Palindrom
   \end{examples}

   \VerticalSpace
   \begin{solution}
      Dies ist die eigentliche Lösung der Aufgabe!
   \end{solution}
   \begin{correction}
      Einen Punkt für das Nennen von Pinguinen!
   \end{correction}
   \subtask{2.5} \label{tsk:exampletask}Diese Teilaufgabe hat nur zwei-ein-halb Punkte. Der Text kann beliebig lang sein. Hier geben wir den Studierenden vier Antwortmöglichkeiten, von denen eine richtig ist:
   \begin{radioboxes}
      \item    Antwortmöglichkeit 1
      \item    Antwortmöglichkeit 2
      \correct Antwortmöglichkeit 3
      \item    Antwortmöglichkeit 4
   \end{radioboxes}
   Wir können auch feste Abstände einbauen:
   \VerticalSpace[2cm]
   \subtask{3} Dies ist erneut eine Aufgabe mit drei Punkten und Antwortmöglichkeiten. Hier können nun aber mehrere Antworten richtig sein:
   \begin{checkboxes}
      \correct Antwortmöglichkeit 1
      \item    Antwortmöglichkeit 2
      \correct Antwortmöglichkeit 3
      \correct Antwortmöglichkeit 4
   \end{checkboxes}
   \subtask{0} Diese Aufgabe gibt zwar keine Punkte (bekommt demnach rechts auch keine Box), wir erwarten von den Studierenden aber trotzdem eine Antwort. Zudem referenzieren wir \autoref{tsk:exampletask} und \autoref{ex:second} nur zum Spaß!

   Die \StudentLine{Laterne auf der Pinguininsel} scheint \StudentLine[2cm]{mittags}!
   % wir können die Abstände beliebig kontrollieren
   \qquad\strut
\end{subtasks}
\end{task}