\documentclass{sp-exam}

\examnumber{1} % 1. Klausur
\examname{Grundlagen der praktischen Informatik}
\examiner{Anonymous Examiner~A, Anonymous Examiner~B, Prof. Unknown}

% date and time of the exam
\semester{Winter}
\date{12. März 2024}
\time{9:00 Uhr}
\duration{90} % in minutes
\year{2023/24}

% hints on the exam cover page
\permittedmaterial{Sie dürfen ein beidseitig, handbeschriebenes DIN-A4-Blatt verwenden.}
% \morehints{
%    \item Consider to avoid cheating, please!
% }
\framedcoverimage{img/nilpferd}

\begin{document}
   \maketitle

   \begin{Exercise}{Eine spaßige Aufgabe}
\begin{tasks}
   % \item Hello % this would cause an error (as a safe-guard)
   \task{3} Dies ist eine Teilaufgabe die drei Punkte wert ist. Für die Antwort geben wir den Studierenden (dynamisch viel) freien Platz!
   \VerticalSpace
   \task{2.5} \label{tsk:exampletask}Diese Teilaufgabe hat nur zwei-ein-halb Punkte. Der Text kann beliebig lang sein. Hier geben wir den Studierenden vier Antwortmöglichkeiten, von denen eine richtig ist:
   \begin{radioboxes}
      \item Antwortmöglichkeit 1
      \item Antwortmöglichkeit 2
      \item Antwortmöglichkeit 3
      \item Antwortmöglichkeit 4
   \end{radioboxes}
   \task{3} Dies ist erneut eine Aufgabe mit drei Punkten und Antwortmöglichkeiten. Hier können nun aber mehrere Antworten richtig sein:
   \begin{checkboxes}
      \item Antwortmöglichkeit 1
      \item Antwortmöglichkeit 2
      \item Antwortmöglichkeit 3
      \item Antwortmöglichkeit 4
   \end{checkboxes}
   \task{0} Diese Aufgabe gibt zwar keine Punkte, wir erwarten von den Studierenden aber trotzdem eine Antwort. Zudem referenzieren wir~\autoref{tsk:exampletask}.

   Die \StudentLine{Laterne auf der Pinguininsel} scheint mittags!
   % wir können die Abstände beliebig kontrollieren
   \qquad\strut
\end{tasks}
\end{Exercise}
   % we can define an exercise without subtasks this way
\begin{exercise}[5]{Hehe!}
\label{ex:second}%
% \Subtask{3}% this would be forbidden and throw an error when running, as you gave
% a fixed amount of points with the exercise ([5])

Für diese Aufgabe benötigen wir Code!
Betrachten Sie bitte dieses ausgezeichnete Java-Programm:

\begin{minted}{java}
public class Main {
   public static void main(String[] args) {
      System.out.println("Hello World!");
   }
}
\end{minted}

\ifexam
Warum auch immer zeigen wir zudem, nur in der eigentlichen Klausur den folgenden Code (z.B. um einfach Code mit und ohne Lücken zu trennen):

\begin{minted}[numbered]{java}
public class Foo {
   public static void main(String[] args) {
      System.out.println("Hi!");
   }
}
\end{minted}
\bigskip

\IndentGuides{9cm}

\IndentGuidesDistance{0.5cm}
\IndentGuides[4]{5cm}

\fi


Inline code? Inline code!

\java{System.out.println("Hello Java!");}\\
\ts{console.log("Hello TypeScript!");}\\
\js{console.log("Hello JavaScript!");}\\
\haskell{main = putStrLn "Hello Haskell!"}\\
\prolog{main :- write('Hello Prolog!'), nl.}\\
\css{body:before { content: "Hello CSS"; }}\\
\xml{<hello>XML</hello>}\\
\html{<html><body>Hello HTML!</body></html>}

\begin{solution}
\begin{minted}{java}
public class Solution {
   public static void main(String[] args) {
      System.out.println("Solution");
   }
}
\end{minted}
\end{solution}


\end{exercise}


   % if we are to lazy, we can specify exercises in-place as well
   \begin{Exercise}{Ho!}
      \Subtask{1}Theoretisch können beliebig viele Teilaufgaben auch manuell definiert werden und so - in welcher form auch immer - in die Aufgabe eingewoben.

      \Subtask*{2}Die Variante mit Stern wird rechts nicht angezeigt.

      \Subtask{3}Allgemein reicht es, \string\Subtask\space zu verwenden.
   \end{Exercise}

   \begin{Exercise}[1]{Witz!}
   \end{Exercise}
   \appendix
   Bar
\end{document}