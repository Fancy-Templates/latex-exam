\documentclass[code]{sp-exam}

\examnumber{1} % 1. Klausur
\examname{Grundlagen der praktischen Informatik}
\examiner{Anonymous Examiner~A, Anonymous Examiner~B, Prof. Unknown}

% date and time of the exam
\semester{Winter}
\date{12. März 2024}
\time{9:00 Uhr}
\duration{90} % in minutes
\year{2023/24}

% hints on the exam cover page
\permittedmaterial{Sie dürfen ein beidseitig, handbeschriebenes DIN-A4-Blatt verwenden.}
% \morehints{
%    \item Consider to avoid cheating, please!
% }
\framedcoverimage{img/nilpferd}

\begin{document}
   \maketitle

   \begin{Exercise}{A Funny Exercise!}
\begin{tasks}
   % \item Hello % this would cause an error (as a safe-guard)
   \task{3} This is a subtask with 3 points.
   \task{2.5} This is a subtask with 2.5 points. It can be arbitrarily long if you want that!
   \newpage
   \task{3} This is a subtask with 3 points (again).
   \task{0} Actually, this subtask has no points at all
\end{tasks}
\end{Exercise}
   % we can define an exercise without subtasks this way
\begin{Exercise}[5]{Hehe!}
\label{ex:second}%
% \Subtask{3}% this would be forbidden and throw an error when running, as you gave
% a fixed amount of points with the exercise ([5])
This is a great exercise!
\end{Exercise}


   % if we are to lazy, we can specify exercises in-place as well
   \begin{exercise}{Ho!}
      \Subtask{1}Theoretisch können beliebig viele Teilaufgaben auch manuell definiert werden und so - in welcher form auch immer - in die Aufgabe eingewoben.

      \Subtask*{2}Die Variante mit Stern wird rechts nicht angezeigt.

      \Subtask{3}Allgemein reicht es, \string\Subtask\space zu verwenden.
   \end{exercise}

   \begin{exercise}[1]{Witz!}
   \end{exercise}

   \appendix

   Bar
   \clearpage
   Foo
\end{document}