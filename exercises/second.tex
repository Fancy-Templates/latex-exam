% we can define an exercise without subtasks this way
\begin{exercise}[5]{Hehe!}
\label{ex:second}%
% \Subtask{3}% this would be forbidden and throw an error when running, as you gave
% a fixed amount of points with the exercise ([5])

Für diese Aufgabe benötigen wir Code!
Betrachten Sie bitte dieses ausgezeichnete Java-Programm:

\begin{minted}{java}
public class Main {
   public static void main(String[] args) {
      System.out.println("Hello World!");
   }
}
\end{minted}

% \ifinmode{exam}
\begin{examonly}
Warum auch immer zeigen wir zudem, nur in der eigentlichen Klausur den folgenden Code (z.B. um einfach Code mit und ohne Lücken zu trennen):
\begin{minted}{java}
public class Foo {
   public static void main(String[] args) {
      System.out.println("Hi!");
   }
}
\end{minted}
\bigskip

\CodeLines{9cm}
\end{examonly}

\begin{solution}
\begin{minted}{java}
public class Solution {
   public static void main(String[] args) {
      System.out.println("Solution");
   }
}
\end{minted}
\end{solution}

\ifinmode{solution,correction}
Much more flexible would be to build stuff up manually with the \string\ifinmode\space command.

\begin{solutionbox}
   Have some fun with magic!

\begin{minted}{haskell}
main :: IO ()
main = putStrLn "Hello World!"
\end{minted}
\end{solutionbox}
\fi

\end{exercise}
