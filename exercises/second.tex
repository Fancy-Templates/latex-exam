% we can define an exercise without subtasks this way
\begin{exercise}[5]{Hehe!}
\label{ex:second}%
% \Subtask{3}% this would be forbidden and throw an error when running, as you gave
% a fixed amount of points with the exercise ([5])

Für diese Aufgabe benötigen wir Code!
Betrachten Sie bitte dieses ausgezeichnete Java-Programm:

\begin{minted}{java}
public class Main {
   public static void main(String[] args) {
      System.out.println("Hello World!");
   }
}
\end{minted}

\ifexam
Warum auch immer zeigen wir zudem, nur in der eigentlichen Klausur den folgenden Code (z.B. um einfach Code mit und ohne Lücken zu trennen):

\begin{minted}[numbered]{java}
public class Foo {
   public static void main(String[] args) {
      System.out.println("Hi!");
   }
}
\end{minted}
\bigskip

\IndentGuides{9cm}

\IndentGuidesDistance{0.5cm}
\IndentGuides[4]{5cm}

\fi


Inline code? Inline code!

\java{System.out.println("Hello Java!");}\\
\ts{console.log("Hello TypeScript!");}\\
\js{console.log("Hello JavaScript!");}\\
\haskell{main = putStrLn "Hello Haskell!"}\\
\prolog{main :- write('Hello Prolog!'), nl.}\\
\css{body:before { content: "Hello CSS"; }}\\
\xml{<hello>XML</hello>}\\
\html{<html><body>Hello HTML!</body></html>}

\begin{solution}
\begin{minted}{java}
public class Solution {
   public static void main(String[] args) {
      System.out.println("Solution");
   }
}
\end{minted}
\end{solution}


\end{exercise}
